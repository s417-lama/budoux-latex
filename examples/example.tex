\documentclass{ltjsarticle}
\usepackage{budoux}

\pagenumbering{gobble}

\begin{document}

{
  \fontsize{38pt}{42pt}\selectfont
  私はその人を常に先生と呼んでいた。だからここでもただ先生と書くだけで本名は打ち明けない。これは世間を憚かる遠慮というよりも、その方が私にとって自然だからである。私はその人の記憶を呼び起すごとに、すぐ「先生」といいたくなる。筆を執っても心持は同じ事である。よそよそしい頭文字などはとても使う気にならない。
}

\clearpage

{
  \fontsize{38pt}{42pt}\selectfont
  \budoux{私はその人を常に先生と呼んでいた。だからここでもただ先生と書くだけで本名は打ち明けない。これは世間を憚かる遠慮というよりも、その方が私にとって自然だからである。私はその人の記憶を呼び起すごとに、すぐ「先生」といいたくなる。筆を執っても心持は同じ事である。よそよそしい頭文字などはとても使う気にならない。}
}

\clearpage

{
  \fontsize{38pt}{42pt}\selectfont
  \budoux[\centering]{私はその人を常に先生と呼んでいた。だからここでもただ先生と書くだけで本名は打ち明けない。これは世間を憚かる遠慮というよりも、その方が私にとって自然だからである。私はその人の記憶を呼び起すごとに、すぐ「先生」といいたくなる。筆を執っても心持は同じ事である。よそよそしい頭文字などはとても使う気にならない。}
}

\clearpage

{
  \fontsize{38pt}{42pt}\selectfont
  \budoux[\raggedleft]{私はその人を常に先生と呼んでいた。だからここでもただ先生と書くだけで本名は打ち明けない。これは世間を憚かる遠慮というよりも、その方が私にとって自然だからである。私はその人の記憶を呼び起すごとに、すぐ「先生」といいたくなる。筆を執っても心持は同じ事である。よそよそしい頭文字などはとても使う気にならない。}
}

\clearpage

{
  \fontsize{38pt}{42pt}\selectfont
  \budoux{私はその人を常に先生と呼んでいた。だからここでもただ先生と書くだけで本名は打ち明けない。これは世間を憚かる遠慮というよりも、その方が私にとって自然だからである。私はその人の記憶を呼び起すごとに、すぐ\textbf{「先生」}といいたくなる。筆を執っても心持は同じ事である。よそよそしい頭文字などはとても使う気にならない。}
}

\clearpage

{
  \fontsize{38pt}{42pt}\selectfont
  \budoux{\textbf{私はその人を常に先生と呼んでいた。だからここでもただ先生と書くだけで本名は打ち明けない。これは世間を憚かる遠慮というよりも、その方が私にとって自然だからである。}私はその人の記憶を呼び起すごとに、すぐ「先生」といいたくなる。筆を執っても心持は同じ事である。よそよそしい頭文字などはとても使う気にならない。}
}

\clearpage

{
  \fontsize{38pt}{42pt}\selectfont
  \budoux{
    特徴量として文字の N グラム(特定の文字が連続して出現するまとまり)のみを使うため、任意の言語に対してモデルを学習、適用できます(Language Neutral)。
  }
}

\clearpage

{
  \fontsize{38pt}{42pt}\selectfont
  \budoux{
    BudouX はこのように \textbf{Small, Standalone, Language Neutral} の 3 つを基本的な方針として開発を進めています。
  }
}

\end{document}
